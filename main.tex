% degree=[master]                             % 必选,学位类型
% language=[chinese|english],                 % 可选(默认:chinese),论文的主要语言
% review=[true|false]					      % 可选,盲审,默认为false
% bibstyle=[gb7714-2015|gb7714-2015ay|ieee],  % 可选(默认:gb7714-2015),参考文献样式
\documentclass[degree=master, language=chinese, review=false, oneside, AutoFakeBold]{shmtuthesis}
\usepackage{xeCJK}
% 所有其它可能用到的包都统一放到shmtuthesis.sty中,可以根据自己的实际添加或者删除。
\usepackage{shmtuthesis}

% 定义图片文件目录与扩展名
\graphicspath{{figures/}}
\DeclareGraphicsExtensions{.pdf, .eps, .png, .jpg, .jpeg}

% 导入参考文献数据库
\addbibresource{thesis.bib}

% 论文信息,必须
\title{上海海事大学学位论文 \LaTeX{} 模板示例文档}
\entitle{A Sample Document for \LaTeX-based SHMTU Thesis Template}
\degreecategory{专业学位}
\author{王群}
\enauthor{Qun Wang}
\studentid{202230311326}
\supervisor{韩德志}
\ensupervisor{Dezhi Han}
\major{软件工程}
\enmajor{Software Engineering}
\finisheddate{二〇二五年四月}

\begin{document}
	
	% 无编号内容:论文封面、授权页
	\maketitle
	\makeDeclareOriginality
	\makeDeclareAuthorization
	
	% 使用罗马数字对前言编号
	\frontmatter
	
	% 摘要
	\input{tex/abstract}
	
	% 目录、插图目录、表格目录、算法目录
	\tableofcontents
	\listoffigures
	\listoftables
	\listofalgorithms
	
	% 使用阿拉伯数字对正文编号
	\mainmatter
	
	% 正文内容
	\input{tex/introduction}
	\chapter{浮动体}

\section{插图}

插图功能是利用 \TeX\ 的特定编译程序提供的机制实现的,不同的编译程序支持不同的图形方式。有的同学可能听说“\LaTeX\ 只支持 EPS”,事实上这种说法是不准确的。\XeTeX 可以很方便地插入 EPS、PDF、PNG、JPEG、JPG 格式的图片。

一般图形都是处在浮动环境中。之所以称为浮动是指最终排版效果图形的位置不一定与源文件中的位置对应,这也是刚使用 \LaTeX\ 同学可能遇到的问题。如果要强制固定浮动图形的位置,请使用 \pkg{float} 宏包,它提供了 \texttt{[H]} 参数。

\subsection{单个图形}

图要有图注,并置于图的编号之后,图的编号和图注应置于图下方的居中位置。引用图应在图题右上角标出文献来源。当插图中组成部件由数字或字母等编号表示时,可在插图下方添加图注进行说明,如图~\ref{fig:shmtu-school-motto} 所示。一般来说,研究生图注与表注一般要求中英文对照。但是由于上海海事大学没有明确要求,故推荐仅使用中文图注。若有需要添加双语图注,用法如图~\ref{fig:shmtu-school-motto-2}所示。

\begin{figure}[!htp]
	\centering
	\includegraphics[width=10cm]{shmtu-school-motto}
	\caption{王伯群校长与吴淞商船校训}
	\label{fig:shmtu-school-motto}
\end{figure}

\begin{figure}[!htp]
	\centering
	\includegraphics[width=10cm]{shmtu-school-motto}
	\bicaption{王伯群校长与吴淞商船校训}{President Wang Boqun and the school motto of WuSong Merchant Shipping}
	\label{fig:shmtu-school-motto-2}
\end{figure}

\subsection{多个图形}

简单插入多个图形的例子如图~\ref{fig:parallel-1} 所示。这两个水平并列放置的子图共用一个图形计数器,没有各自的子图题。

\begin{figure}[!htp]
  \centering
  \includegraphics[height=2cm]{images/shmtu-badge}
  \hspace{1cm}
  \includegraphics[height=2cm]{images/shmtu-badge}
  \caption{上海海事大学校徽}
  \label{fig:parallel-1}
\end{figure}

如果多个图形相互独立,并不共用一个图形计数器,那么用 \cs{minipage} 或者\cs{parbox} 就可以,如图~\ref{fig:parallel-2-1} 与图~\ref{fig:parallel-2-2}。

\begin{figure}[!htp]
  \begin{minipage}{0.48\textwidth}
  	\centering
    \includegraphics[height=1.5cm]{images/shmtu-badge.png}
    \caption{并排第一个图}
    \label{fig:parallel-2-1}
  \end{minipage}
  \hfill
  \begin{minipage}{0.48\textwidth}
    \centering
    \includegraphics[height=1.5cm]{images/shmtu-badge.png}
    \caption{并排第二个图}
    \label{fig:parallel-2-2}
  \end{minipage}
\end{figure}

如果要为共用一个计数器的多个子图添加子图注,那么使用\cs{subcaptionbox}(双语图注使用\cs{bisubcaptionbox})并排子图,子图注置于子图之下,子图号用 (a)、(b)、(c)等表示。如图~\ref{fig:subcaptionbox}、图~\ref{fig:subcaptionbox-a}、图~\ref{fig:subcaptionbox-b}、图~\ref{fig:subcaptionbox-c}所示。

\begin{figure}[!htp]
  \subcaptionbox{上海海事大学校徽\label{fig:subcaptionbox-a}}%
	[0.5\textwidth]{%
	  \includegraphics[height=1.5cm]{images/shmtu-badge}
	}
  \hfill
  \subcaptionbox{上海海事大学校名\label{fig:subcaptionbox-b}}%
    [0.5\textwidth]{%
	  \includegraphics[width=0.5\textwidth]{images/shmtu-name}
	}
  \hfill
  \subcaptionbox{上海海事大学校徽\label{fig:subcaptionbox-c}}%
    [\textwidth]{%
	  \includegraphics[height=1.5cm]{images/shmtu-badge}
  }
  \caption{共用一个计数器的多个子图图注}
  \label{fig:subcaptionbox}
\end{figure}

\section{表格}

\subsection{基本表格}

编排表格应简单明了,表达一致,明晰易懂,表文呼应、内容一致。表注置于表上,研究生学位论文可以用中、英文两种文字居中排写,中文在上,也可以只用中文。

表格的编排采用国际通行的三线表\footnote{三线表,以其形式简洁、功能分明、阅读方便而在科技论文中被推荐使用。三线表通常只有 3 条线,即顶线、底线和栏目线,没有竖线。}。三线表可以使用 \pkg{booktabs} 提供的 \cs{toprule}、\cs{midrule} 和 \cs{bottomrule}。它们与 \pkg{longtable} 能很好的配合使用。

\begin{table}[!htp]
  \centering
  \caption[一个标准的三线表]{一个标准的三线表\footnotemark}
  \label{tab:firstone}
  \begin{tabular}{llr}  
    \toprule
    \multicolumn{2}{c}{Item} \\
    \cmidrule(r){1-2}
    Animal    & Description & Price (\$) \\
    \midrule
    Gnat      & per gram    & 13.65      \\
              &    each     & 0.01       \\
    Gnu       & stuffed     & 92.50      \\
    Emu       & stuffed     & 33.33      \\
    Armadillo & frozen      & 8.99       \\
    \bottomrule
  \end{tabular}
\end{table}
\footnotetext{这个例子来自
  \href{https://mirrors.sjtug.sjtu.edu.cn/ctan/macros/latex/contrib/booktabs/booktabs.pdf}%
  {《Publication quality tables in LaTeX》}(\pkg{booktabs} 宏包的文档)。这也是
  一个在表格中使用脚注的例子,请留意与 \pkg{threeparttable} 实现的效果有何不
  同。}
  
\subsection{复杂表格}

我们经常会在表格下方标注数据来源,或者对表格里面的条目进行解释。可以用\pkg{threeparttable} 实现带有脚注的表格,如表~\ref{tab:footnote}。

\begin{table}[!htpb]
  \bicaption{一个带有脚注的表格的例子}{A Table with footnotes}
  \label{tab:footnote}
  \centering
  \begin{threeparttable}[b]
     \begin{tabular}{ccd{4}cccc}
      \toprule
      \multirow{2}*{total} & \multicolumn{2}{c}{20\tnote{a}} & \multicolumn{2}{c}{40} & \multicolumn{2}{c}{60} \\
      \cmidrule(lr){2-3}\cmidrule(lr){4-5}\cmidrule(lr){6-7}
      & www & \multicolumn{1}{c}{k} & www & k & www & k \\ % 使用说明符 d 的列会自动进入数学模式,使用 \multicolumn 对文字表头做特殊处理
      \midrule
      & $\underset{(2.12)}{4.22}$ & 120.0140\tnote{b} & 333.15 & 0.0411 & 444.99 & 0.1387 \\
      & 168.6123 & 10.86 & 255.37 & 0.0353 & 376.14 & 0.1058 \\
      & 6.761    & 0.007 & 235.37 & 0.0267 & 348.66 & 0.1010 \\
      \bottomrule
    \end{tabular}
    \begin{tablenotes}
      \item [a] the first note.% or \item [a]
      \item [b] the second note.% or \item [b]
    \end{tablenotes}
  \end{threeparttable}
\end{table}

\zhlipsum[1]

如某个表需要转页接排,可以用 \pkg{longtable} 实现。接排时表注省略,表头应重复书写,并在右上方写“续表 xx”,如表~\ref{tab:grid_mlmmh}。

\begin{longtable}[c]{c*{3}{r}}
  \caption[高变网格的可行三元组]{高变网格的可行三元组, MLMMH.}
  \label{tab:grid_mlmmh} \\
  % 表头
  \toprule
  \multicolumn{1}{c}{Time (s)} & \multicolumn{1}{c}{Triple chosen} & \multicolumn{1}{c}{Other feasible triples} \\
  \midrule
  \endfirsthead
	
  % 续表
  \multicolumn{3}{r}{续表~\thetable} \\
  \toprule
  % 续表表头
  \multicolumn{1}{c}{\textbf{Time (s)}} & \multicolumn{1}{c}{\textbf{Triple chosen}} & \multicolumn{1}{c}{\textbf{Other feasible triples}} \\ 
  \midrule
  \endhead

  \midrule
  \multicolumn{3}{r}{续下页} \\
  \endfoot
  
  \bottomrule
  \endlastfoot
  
0 & (1, 11, 13725) & (1, 12, 10980), (1, 13, 8235), (2, 2, 0), (3, 1, 0) \\
2745 & (1, 12, 10980) & (1, 13, 8235), (2, 2, 0), (2, 3, 0), (3, 1, 0) \\
5490 & (1, 12, 13725) & (2, 2, 2745), (2, 3, 0), (3, 1, 0) \\
8235 & (1, 12, 16470) & (1, 13, 13725), (2, 2, 2745), (2, 3, 0), (3, 1, 0) \\
10980 & (1, 12, 16470) & (1, 13, 13725), (2, 2, 2745), (2, 3, 0), (3, 1, 0) \\
13725 & (1, 12, 16470) & (1, 13, 13725), (2, 2, 2745), (2, 3, 0), (3, 1, 0) \\
16470 & (1, 13, 16470) & (2, 2, 2745), (2, 3, 0), (3, 1, 0) \\
19215 & (1, 12, 16470) & (1, 13, 13725), (2, 2, 2745), (2, 3, 0), (3, 1, 0) \\
21960 & (1, 12, 16470) & (1, 13, 13725), (2, 2, 2745), (2, 3, 0), (3, 1, 0) \\
24705 & (1, 12, 16470) & (1, 13, 13725), (2, 2, 2745), (2, 3, 0), (3, 1, 0) \\
27450 & (1, 12, 16470) & (1, 13, 13725), (2, 2, 2745), (2, 3, 0), (3, 1, 0) \\
30195 & (2, 2, 2745) & (2, 3, 0), (3, 1, 0) \\
32940 & (1, 13, 16470) & (2, 2, 2745), (2, 3, 0), (3, 1, 0) \\
35685 & (1, 13, 13725) & (2, 2, 2745), (2, 3, 0), (3, 1, 0) \\
38430 & (1, 13, 10980) & (2, 2, 2745), (2, 3, 0), (3, 1, 0) \\
41175 & (1, 12, 13725) & (1, 13, 10980), (2, 2, 2745), (2, 3, 0), (3, 1, 0) \\
43920 & (1, 13, 10980) & (2, 2, 2745), (2, 3, 0), (3, 1, 0) \\
46665 & (2, 2, 2745) & (2, 3, 0), (3, 1, 0) \\
49410 & (2, 2, 2745) & (2, 3, 0), (3, 1, 0) \\
52155 & (1, 12, 16470) & (1, 13, 13725), (2, 2, 2745), (2, 3, 0), (3, 1, 0) \\
54900 & (1, 13, 13725) & (2, 2, 2745), (2, 3, 0), (3, 1, 0) \\
57645 & (1, 13, 13725) & (2, 2, 2745), (2, 3, 0), (3, 1, 0) \\
60390 & (1, 12, 13725) & (2, 2, 2745), (2, 3, 0), (3, 1, 0) \\
63135 & (1, 13, 16470) & (2, 2, 2745), (2, 3, 0), (3, 1, 0) \\
65880 & (1, 13, 16470) & (2, 2, 2745), (2, 3, 0), (3, 1, 0) \\
68625 & (2, 2, 2745) & (2, 3, 0), (3, 1, 0) \\
71370 & (1, 13, 13725) & (2, 2, 2745), (2, 3, 0), (3, 1, 0) \\
74115 & (1, 12, 13725) & (2, 2, 2745), (2, 3, 0), (3, 1, 0) \\
76860 & (1, 13, 13725) & (2, 2, 2745), (2, 3, 0), (3, 1, 0) \\
79605 & (1, 13, 13725) & (2, 2, 2745), (2, 3, 0), (3, 1, 0) \\
82350 & (1, 12, 13725) & (2, 2, 2745), (2, 3, 0), (3, 1, 0) \\
85095 & (1, 12, 13725) & (1, 13, 10980), (2, 2, 2745), (2, 3, 0), (3, 1, 0) \\
87840 & (1, 13, 16470) & (2, 2, 2745), (2, 3, 0), (3, 1, 0) \\
\end{longtable}


\section{算法环境}

算法环境可以使用 \pkg{algorithms} 宏包或者较新的 \pkg{algorithm2e} 实现。算法~\ref{algo:algorithm} 是一个使用 \pkg{algorithm2e} 的例子。关于排版算法环境的具体方法,请阅读相关宏包的官方文档\footnote{\url{http://tug.ctan.org/macros/latex/contrib/algorithm2e/doc/algorithm2e.pdf}}。

\begin{algorithm}
  \DontPrintSemicolon % Some LaTeX compilers require you to use \dontprintsemicolon instead
  \KwIn{A finite set $A=\{a_1, a_2, \ldots, a_n\}$ of integers}
  \KwOut{The largest element in the set}
  
  $max \gets a_1$\;
  \For{$i \gets 2$ \textbf{to} $n$} {
    \eIf{$a_i > max$} {
      $max \gets a_i$\;
    }{
      pass\;
    }
  }
  \Return{$max$}\;
  \caption{{\sc Max} finds the maximum number}
  \label{algo:algorithm}
\end{algorithm}

\section{代码环境}

我们可以在论文中插入算法,但是不建议插入大段的代码。如果确实需要插入代码,建议使用 \pkg{listings} 宏包。

\begin{codeblock}[language=Python]
# -*- coding: utf-8 -*-
import click

from app.extensions import db
from app.models import Role


def register_commands(app):
    @app.cli.command()
    @click.option('--drop', is_flag=True, help='删除之前的表后再初始化.')
    def initdb(drop):
        """初始化数据库."""
        if drop:
            click.confirm('执行该命令将会删除当前数据库,确定要执行吗?', abort=True)
            db.drop_all()
            click.echo('删除所有表.')
        db.create_all()
        click.echo('初始化数据库.')

    @app.cli.command()
    def init():
        """初始化项目"""
        click.echo('初始化数据库...')
        db.create_all()
        
        click.echo('初始化用户角色与权限...')
        Role.init_role()

        click.echo('初始化完毕.')
\end{codeblock}


	\chapter{数学符号与引用文献的标注}

\section{数学}

\subsection{数字与单位}

宏包 \pkg{siunitx} \footnote{\url{http://tug.ctan.org/macros/latex/exptl/siunitx/siunitx.pdf}}提供了更好的数字和单位支持,具体请查看相关文档:
\begin{itemize}
	\item \num{12345,67890}
	\item \num[parse-numbers=false]{1-2i}
	\item \num{.3e45}
	\item \num[parse-numbers=false]{1.654 x 2.34 x 3.430}
	\item \si{kg.m.s^{-1}}
	\item \si{\kilogram\metre\per\second}
	\item \si[per-mode=symbol]{\kilogram\metre\per\second}
	\item \si[per-mode=symbol]{\kilogram\metre\per\ampere\per\second}
	\item \numlist{10;20;30}
	\item \SIlist{0.13;0.67;0.80}{\milli\metre}
	\item \numrange{10}{20}
    \item \SIrange{10}{20}{\degreeCelsius}
	\item \SIrange{0.13}{0.67}{\milli\metre}
	\item \ang{10}
	\item \ang{12.3}
	\item \ang{4,5}
	\item \ang{1;2;3}
	\item \ang{;;1}
	\item \ang{+10;;}
	\item \ang{-0;1;}
\end{itemize}

\subsection{数学符号和公式}

本小节仅演示基本用法,数学符号、公式、数组的详细内容,请查看文档\footnote{\url{https://en.wikibooks.org/wiki/LaTeX/Mathematics}}。

微分符号 $\dif$ 应使用正体,本模板提供了 \cs{dif} 命令。除此之外,模板还提供了一些命令方便使用:
\begin{itemize}
  \item 圆周率 $\uppi$:\verb|\uppi|
  \item 自然对数的底 $\upe$:\verb|\upe|
  \item 虚数单位 $\upi$, $\upj$:\verb|\upi| \verb|\upj|
\end{itemize}

公式应另起一行居中排版。公式后应注明编号,按章顺序编排,编号右端对齐。

\begin{equation}
	\cos (2\theta) = \cos^2 \theta - \sin^2 \theta
\end{equation}

\begin{equation}
  \frac{\dif^2 u}{\dif t^2} = \int f(x) \dif x.
\end{equation}

公式末尾是需要添加标点符号的,至于用逗号还是句号,取决于公式下面一句是接着公式说的,还是另起一句。

\begin{equation}
	\frac{2h}{\pi}\int_{0}^{\infty}\frac{\sin\left( \omega\delta \right)}{\omega}
	\cos\left( \omega x \right) \dif\omega = 
	\begin{cases}
		h, \ \left| x \right| < \delta, \\
		\frac{h}{2}, \ x = \pm \delta, \\
		0, \ \left| x \right| > \delta.
	\end{cases}
\end{equation}

公式较长时最好在等号“$=$”处转行。
\begin{align}
    & I (X_3; X_4) - I (X_3; X_4 \mid X_1) - I (X_3; X_4 \mid X_2) \nonumber \\
  = & [I (X_3; X_4) - I (X_3; X_4 \mid X_1)] - I (X_3; X_4 \mid \tilde{X}_2) \\
  = & I (X_1; X_3; X_4) - I (X_3; X_4 \mid \tilde{X}_2).
\end{align}

如果在等号处转行难以实现,也可在 $+$、$-$、$\times$、$\div$运算符号处转行,转行时运算符号仅书写于转行式前,不重复书写。
\begin{multline}
  \frac{1}{2} \Delta (f_{ij} f^{ij}) =
    2 \left(\sum_{i<j} \chi_{ij}(\sigma_{i} - \sigma_{j})^{2}
    + f^{ij} \nabla_{j} \nabla_{i} (\Delta f) \right. \\
  \left. + \nabla_{k} f_{ij} \nabla^{k} f^{ij} +
    f^{ij} f^{k} \left[2\nabla_{i}R_{jk}
    - \nabla_{k} R_{ij} \right] \vphantom{\sum_{i<j}} \right).
\end{multline}


需要在文中引用某个指定公式,如公式~\ref{eq:array}所示:

\begin{equation}
	A_{m,n} = 
	\begin{pmatrix}
		a_{1,1} & a_{1,2} & \cdots & a_{1,n} \\
		a_{2,1} & a_{2,2} & \cdots & a_{2,n} \\
		\vdots  & \vdots  & \ddots & \vdots  \\
		a_{m,1} & a_{m,2} & \cdots & a_{m,n} 
	\end{pmatrix}\label{eq:array}
\end{equation}

\subsection{定理环境}

示例文件中使用 \pkg{ntheorem} 宏包配置了定理、引理和证明等环境。用户也可以使用\pkg{amsthm} 宏包。

这里举一个“定理”和“证明”的例子:
\begin{theorem}[留数定理]\label{thm:res}
  假设 $U$ 是复平面上的一个单连通开子集,$a_1, \ldots, a_n$ 是复平面上有限个点,
  $f$ 是定义在 $U \backslash \{a_1, \ldots, a_n\}$ 上的全纯函数,如果 $\gamma$
  是一条把 $a_1, \ldots, a_n$ 包围起来的可求长曲线,但不经过任何一个 $a_k$,并且
  其起点与终点重合,那么:

  \begin{equation}
    \label{eq:res}
    \ointop_\gamma f(z)\, \dif z = 2\uppi \upi \sum_{k=1}^n \operatorname{I}(\gamma, a_k) \operatorname{Res}(f, a_k).
  \end{equation}

  如果 $\gamma$ 是若尔当曲线,那么 $\operatorname{I}(\gamma, a_k) = 1$,因此:

  \begin{equation}
    \label{eq:resthm}
    \ointop_\gamma f(z)\, \dif z = 2\uppi \upi \sum_{k=1}^n \operatorname{Res}(f, a_k).
  \end{equation}

  在这里,$\operatorname{Res}(f, a_k)$ 表示 $f$ 在点 $a_k$ 的留数,
  $\operatorname{I}(\gamma, a_k)$ 表示 $\gamma$ 关于点 $a_k$ 的卷绕数。卷绕数是
  一个整数,它描述了曲线 $\gamma$ 绕过点 $a_k$ 的次数。如果 $\gamma$ 依逆时针方
  向绕着 $a_k$ 移动,卷绕数就是一个正数,如果 $\gamma$ 根本不绕过 $a_k$,卷绕数
  就是零。
\end{theorem}

定理~\ref{thm:res} 的证明。
	
\begin{proof}
  首先,由\dots

  其次,\dots

  所以,\dots
\end{proof}

\section{引用文献的标注}

按照上海海事大学的要求,参考文献外观应符合国标 GB/T 7714 的要求。具体建议使用87版标准,由于这个版本太老(1988年1月1日实施),故本模版使用该国标下最新的2015版标准。本模版使用 \BibLaTeX\ 配合 \pkg{biblatex-gb7714-2015} 样式包\footnote{\url{https://www.ctan.org/pkg/biblatex-gb7714-2015}}控制参考文献的输出样式,后端采用 \pkg{biber} 管理文献。

请注意 \pkg{biblatex-gb7714-2015} 宏包 2016 年 9 月才加入 CTAN,如果你使用的\TeX\ 系统版本较旧,可能没有包含 \pkg{biblatex-gb7714-2015} 宏包,需要手动安装。\BibLaTeX\ 与 \pkg{biblatex-gb7714-2015} 目前在活跃地更新,为避免一些兼容性问题,推荐使用较新的版本。

正文中引用参考文献时,使用 \verb|\cite{key1,key2,key3...}| 可以产生“上标引用的参考文献”。使用\verb|\parencite{key1, key2, key3...}| 则可以产生水平引用的参考文献。建议将bibtex文献中的标示都改为英文,以免出现不兼容现象。

具体请看下面的例子,将会穿插使用水平的和上标的参考文献:Chen调查了用于语言n-gram建模的平滑模型的最广泛使用的算法,并提出了改进的语言模型平滑度,从而改善了语音识别性能\cite{chen1999empirical};SRILM是C ++库,可执行程序和帮助程序脚本的集合,旨在允许为语音识别和其他应用程序生成统计语言模型并进行实验\cite{stolcke2002srilm}。Sundermeyer、Soutner、王毅、梁军等人将LSTM应用到自然语言处理领域,并获得了不错的实验结果\cite{sundermeyer2012lstm, soutner2013application, wangyi2018, liangjun2015}。文献\parencite{sundermeyer2012lstm, soutner2013application, wangyi2018, liangjun2015}中均使用LSTM神经网络架构。

当需要将参考文献条目加入到文献表中但又不在正文中引用,可以使用\verb|\nocite{key1,key2,key3...}|。或者使用 \verb|\nocite{*}| 将参考文献数据库中的所有条目加入到文献表中。\nocite{*}

	\input{tex/summary}
	
	% 致谢
	\input{tex/acknowledgements}
	
	% 参考文献
	\printbibliography[heading=bibintoc]
	
	% 使用英文字母对附录编号
	\appendix
	\input{tex/appendix/maxwell_equations}
	\input{tex/appendix/flow_chart}
	
	% 文后无编号部分
	\backmatter
	
	% 发表论文、获奖情况、申请专利、参与项目
	% 盲审论文中,发表学术论文及参与科研情况等仅以第几作者注明即可,不要出现作者或他人姓名
	\input{tex/achievements/publications}
	\input{tex/achievements/awards}
	\input{tex/achievements/patents}
	\input{tex/achievements/projects}
	
\end{document}